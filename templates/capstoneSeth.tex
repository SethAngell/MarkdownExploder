
%
% thesis.tex
%
% Modified from the Clemson Template
% Master's Thesis/Ph.D. Dissertation Template
% Clemson University
%
%
% The document guidelines say the font can be between 10pt and 12pt.
% Specify whatever you want it to be here.
%
\documentclass[10pt]{templates/ClemsonThesis}

%
% Use any additional packages you might need
%
%\usepackage[table]{xcolor}
%\documentclass[xcolor=table]{beamer}
%\usepackage[table]{xcolor}

\usepackage{graphicx}
\usepackage{pdfpages}
\usepackage{multirow}
\usepackage{caption}
\usepackage{subcaption}
\usepackage[normalem]{ulem}
\useunder{\uline}{\ul}{}
% Please add the following required packages to your document preamble:
%\usepackage{multirow}
%\usepackage{colortbl}
%% If you use beamer only pass "xcolor=table" option, i.e. \documentclass[xcolor=table]{beamer}
% \usepackage[normalem]{ulem}
% \useunder{\uline}{\ul}{}

%
% Make the document your own -- fill in these values to reflect the type of
% document you are writing.
%
\title{Scalable Session Monitoring: A Unity3D Framework Proposal}
\department{Cameron School of Business}
\documentType{Capstone}
\major{Computer Science}
\degree{Masters of Science in Computer Science and Information Systems}
\graduationMonth{May}
\graduationYear{2022}
\author{Seth Angell}
\committeeChair{Dr. Toni Pence}
\committeeMemberOne{Dr. Ron Vetter}
\committeeMemberTwo{Dr. Douglas Kline}

%
% PDF Setup -- most of this you do not need to touch
%
\hypersetup{
    colorlinks,
    linkcolor={black},
    citecolor={black},
    filecolor={black},
    urlcolor={black},
    pdftitle={\theTitle},
    pdfauthor={\theAuthor},
    pdfsubject={\theDocumentType},
    pdfkeywords={University of North Carolina Wilmington, \theDepartment, \theDocumentType, \theMajor, \theDegree},
    pdfstartpage={1},
}


%
% User-specified command definitions/redefinitions
%
%% \newcommand{\cplusplus}{{\rm C\raise.5ex\hbox{\small ++}}}
%% \newcommand{\num}[1]{\mbox{(\textit{#1})}}
%% \renewcommand{\ttdefault}{pcr}
%% \renewcommand\lstlistlistingname{List of Listings}

\usepackage[justification=centering]{caption}

\begin{document}
%  ============================================================================
    \frontmatter % Begin front matter (pages are numbered with Roman numerals)
%  ============================================================================

    \addtotoc{Title Page}{\maketitle}          % Generate the title page
    \doublespacing                             % Text should be double spaced
    \setcounter{page}{2}                       % Abstract begins on page 2
    % \addtotoc{Abstract}{\input{abstract.tex}}  % Generate the abstract

    %
    % The dedication page is optional.  Comment out this line if you do not
    % want to include this page.
    %
    %\addtotoc{Dedication}{\input{dedication.tex}}

    %
    % The acknowledgment page is optional.  Comment out this line if you do
    % not want to include this page.
    %
    %\addtotoc{Acknowledgments}{\input{acknowledgments.tex}}

    \singlespacing                             % Single space the lists
    \tableofcontents \clearpage                % Generate the Table of Contents

    %
    % REMEMBER: Review your caption listings in the genrated lists
    %           and make sure they include '\newline' commands as necessary.
    %           See the README for further information.
    %
    %\addtotoc{List of Tables}{\listoftables}   % Generate the List of Tables
    %\addtotoc{List of Figures}{\listoffigures} % Generate the List of Figures

    %
    % Include other optional lists.  Computer science, for example, would
    % likely include a 'List of Listings' (and would \usepackage{listings}
    % and \renewcommand\lstlistlistingname{List of Listings}).
    %
    %% \addtotoc{List of Listings}{\lstlistoflistings}



%  ===========================================================================
    \mainmatter % Begin main matter (pages are numbered with Arabic numerals)
%  ===========================================================================
    \doublespacing % Text should be double spaced

    %
    % Here we have each chapter in a separate file.  Name these as you choose,
    % and include them in the order you want them to appear.  Be sure to use
    % the \inputfile command.
    %
    \inputfile{staging/introduction.tex}
    \inputfile{staging/background.tex}
    \inputfile{staging/proposal.tex}
    \inputfile{staging/predictions.tex}
    \inputfile{staging/testing_plan.tex}
    \inputfile{staging/results.tex}
    \inputfile{staging/conclusion.tex}
    % \inputfile{results.tex}
    %\inputfile{chapter_experiment4.tex}	
    %\inputfile{chapter_experiment5.tex}
    %
    % The appendices are optional.  This is the format for two or more.
    % If you do not wish to include an appendix, comment out these lines.
    % If you want just one, see the formatting guidelines.
    %
%     \begin{appendices}
%         \begin{subappendices}
%         \inputfile{glossary.tex}
% %         \inputfile{appendixB.tex}
% %         \inputfile{appendixC.tex}
% %	  \inputfile{appendixD.tex}
% %	  \inputfile{appendixE.tex}
%         \end{subappendices}
%     \end{appendices}



    \singlespacing                             % Single space the Bibliography

  
   \bibliographystyle{apalike}
   \addtotoc{Bibliography}{\bibliography{works_cited}}

\end{document}
